\documentclass[12 pt]{article}
\pagestyle{empty}

\usepackage{amssymb}
\usepackage{mathrsfs}
\usepackage{amsmath}
\usepackage{verbatim}
\usepackage{color}
\usepackage{ulem}
\usepackage[pdftex]{graphicx}     
\graphicspath{{../pdf/}{../jpeg/}{./image/}}    
\DeclareGraphicsExtensions{.pdf,.jpeg,.png,.jpg}   

%One-Inch Margins, 8 1/2 X 11 Paper
\setlength{\oddsidemargin}{0 in}
\setlength{\evensidemargin}{0 in}
\setlength{\textwidth}{6.5 in}

\setlength{\topmargin}{-0.75 in}
\setlength{\footskip}{0 in}
\setlength{\textheight}{9 in}

\begin{document}

\begin{center}
	SRM 635 Notes\\
	\color{blue}
	Matt Swaffer Fall 2018
	\color{black}
\end{center}

\begin{center}
    9/4/2018
\end{center}
\section{Sampling}
\begin{enumerate}
    \item Poisson Sampling
        \begin{itemize}
            \item None of the totals are fixed
            \item Independent variables
        \end{itemize}
    \item Multinomial Sampling
        \begin{itemize}
            \item Total sample size is fixed (i.e., quota for study)
            \item Each cell is a component of a single multinomial (the total)
            \item Considered a single independent observation
        \end{itemize}
    \item Independent multinomial sampling
        \begin{itemize}
            \item Fix totals for one variable
            \item Each of the multinomials is an independent observation
        \end{itemize}
    \item Hypergeometric sampling
        \begin{itemize}
            \item All totals are fixed
            \item Fisher's exact test
            \item Control both rows and columns
        \end{itemize}
\end{enumerate}


\section{Hypothesis Tests} 
\begin{itemize} 
    \item[1] Tests of Independence
        \begin{itemize}
        \item [a.] Joint = Product of marginals
        \item [b.] You can separate probability into both rows and columns (only applicable to Poisson or Multinomial.. rows and columns must be random. Cannot have fixed rows or columns.)
        \item [c.] Pearson's \( \chi ^2\) 
            \begin{enumerate}
                \item Assume Poisson sampling
                \item Each cell is a Poisson random variable
                \item Construct a Wald statistic for each cell\\
                    You end up with a \(\chi ^2\) for each cell \\
                    Sum all the \( \chi ^2 \) across the table to get a single value. 
            \end{enumerate}
        \item [b.]
        \item [d.] Likelihood ratio test
            \begin{itemize}
                \item Idea: Compare values of the likelihood function under Null and alternative
                \item Want to know if the additional parameters of the alternative are meaningful
                $$ G^2=2 \sum_{i,j} n_{ij} ln \Big( \frac{n_{ij}n_{++}}{n_{i+}n_{+j}} \Big) $$
                Same distribution, DF as Pearson \(X^2\)
                \item Test statistic
                $$ ln \Big( \frac{l_0}{l_a} \Big) $$
            \end{itemize} 
        
        \item [e.] Residual analysis
            \begin{itemize}
                \item Not used for checking assumptions
                \item Used as post-hoc calculations to identify non-independence
                \item Pearson Residuals
                $$e_{ij} = \frac{cell count - expected}{variation} $$ 
                $$e_{ij} = \frac{n_{ij}-\frac{n_{i+}n_{n+j}}{n_{++}}}{\sqrt{\frac{n_{i+}n_{+j}}{n_{++}}}} $$
                \item These are almost standard normal
                $$e_{ij}~ N(0, \nu)  , \nu < 1$$
                These are conservative
                \item Standardized Pearson residuals
                \item For \( X^2 \) at least 80\% of cells \( E_{ij} \geq 5\)
                \item For \(G^2:  \frac{n_{++}}{IxJ}\geq 5\)
                \item if these fail, use Fischer's Exact Test\\
                Idea: Treat row / column totals as fixed \\
                Identify all possible cell counts that lead to these totals\\
                Count up all those showing "more" dependence than observed\\
                Hypergeometric distribution???\\
                No sample size restriction
            \end{itemize}
       
    \end{itemize}
    \vskip 1 in
    \item [2] Test of Homogeneity
    \vskip 1 in
    \item [3] Tests for Ordinal Data
        \begin{itemize}
        \item Testing hypothesis with ordered alternatives (trends) provides greater power
        \item [a.] Linear Trend Test
            \begin{itemize}
                \item \( H_0: variables are not associated \)
                \item \( H_a: Linear trend in levels of one variable across the levels of another \)
                \item Example: Income level\\
                \begin{tabular}{c|c|c|c|c}
                        & $<$ 20k & 20K - 50K & 50K - 100K & $<$ 100K\\
                    18-24 & & & &\\
                    25-39 &  & & &\\
                     $<$ 40 & & & &\\
                \end{tabular}
                \item LT Test Statistic\\
                Start by assigning weights to all variable levels (ordinal by magnitude\\
                Similar to a heat map \\
                Choose weights: \uwave{u} and \uwave{v}\\
                Use weighted sum:\\
                $ S = \sum_{i,j} n_{ij} u_i v_j $\\\\
                $ Q=\frac{(S-E[s]^2)}{Var(s)} $ \\
                -Essentially weighted correlation\\
                \(Q = M^2 = (n_{++}-1)r^2 = X^2(1)\)\\
                If you know ahead of time, you can perform a one-sided test, Use M, otherwise use \(M^2\) for two-sided\\
                One sided: \(H_0: no trend | H_a: trend\)\\
                Two sided: \(H_0: no trend | H_a: + / - trend\)\\  
                Small p value = "data provided sufficient evidence to conclude that the weight / magnitude / heat of income level counts changed linearly with increased age"
            \end{itemize}
        
    \end{itemize}
    \vskip 1 in
    \item [---] Properties of 3 Categorical Variables 
        \begin{enumerate}
                \item Marginal: sum over 3rd variable
                \item Conditional: Make conclusions for each level of 3rd variable 
            \end{enumerate}
    \vskip 1 in
    \item [---] Marginal Independence \\
        $ \pi_{ij} = \pi_{i++} \pi_{+j+} $
    \vskip 1 in
    \item [4.] Conditional Independence
        \begin{itemize}
        \item Probabilistic definition\\
        $P(Y=i|X=j, Z=k) = P(Y=i|Z=k)$\\
        Y independent of X conditional on Z\\
        \item Joint probability: \( \frac{Product of row, col calc given z}{distribution of z}\)\\
        $ \pi_{ij} = \frac{\pi_{i+k} \pi_{+jk}}{\pi_{++K}}$\\
        Columns are conditionally independent of the rows IF:\\
        $\pi_{ijk} = \frac{\pi_{i+k} \pi_{+jk}}{\pi_{++k}}$ for any k\\
        (In other words... there is no interaction of rows and columns based on the 3rd variable)\\
        \item Conditional Independence: \( \theta_{xy|i} = 1 for every k \)
        \item Cochran - Mantel - Haenszel Test of Conditional independence\\
        \( H_0\): two variables (x and y) are independent of each other conditional on another variable) \\
        \( H_a\): two variables (x and y) are not independent ...\\
        \( H_0: \theta_{xy|z} = 1\)\\
        \( H_a:\) at least one \(\theta_{xy|k} \neq 0\)\\
        \(CMH = X^2(1)\)\\
        small p \\
        "At least one of the cities there is an association between smoking and cancer"\\
        \item [---] Homogenous association
        \begin{itemize}
            \item Weaker than conditional independence
            \item 2 variables can be associated but must be associated equally across levels of z\\
            The variables can be related, but it does not change based on the level of the third variable. (There is no interaction effect)\\
            \item Breslow-Day Test \\
            large p\\
            "fail to reject evidence that association is the same across cities"
        \end{itemize}
        \item Conditional independence implies homogenous association.
        \item The relationship between x and y does not depend on z 
        \item The cell counts are your outcome... everything else is an IV.\\ 
        There is no three-way interaction between x, y and z.
    \end{itemize}

\end{itemize}
\end{document}
